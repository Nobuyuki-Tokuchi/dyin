\documentclass[a4paper,xelatex,ja=standard]{bxjsarticle}

\usepackage{amsmath, amssymb}
\usepackage{ascmac}
\usepackage{hyperref}
\hypersetup{pdfborder={0 0 0},bookmarksopen=true}

\usepackage{liparxe}
\font\charis="Charis SIL"

\begin{document}

{
\title{\Huge 現代ヴィッセンスタンツ語文法}
\author{Skarsna.haltxeafis/渡久地信之}
\date{2017年}
}
\maketitle
\thispagestyle{empty}

% 表紙のページのページ数を消して次からカウントさせるために
% 目次のページの番号をromanにすることで対処している
% 正しいやり方を見付け次第修正が必要かもしれない

\newpage

\pagenumbering{roman}
\tableofcontents

\newpage
\pagenumbering{arabic}

\section{概要}
現代ヴィッセンスタンツ語は,連邦接触前からデュインで話されている言語である.
% 何か書く

\section{文字と発音}
\subsection{基本的な発音}
現代ヴィッセンスタンツ語で使用されている文字とその発音は次のようになっている.

\begin{table}[htbp]
\begin{center}
 \caption{現代ヴィッセンスタンツ語で使用される文字とその発音}
 \label{vis_phonology}
 \begin{tabular}[tb]{|c||c|c|c|c|c|c|c|c|c|c|c|c|c|} \hline
  リパーシェ & \liparxea & \liparxeb & \liparxec & \liparxed & \liparxedz & \liparxee & \liparxef
             & \liparxeg & \liparxeh & \liparxei & \liparxej & \liparxek & \liparxel \\ \hline
  転写 & a & b & c & d & dz & e & f & g & h & i & j & k & l \\ \hline
  音素 & \charis a & \charis b & \charis s & \charis d & \charis \textyogh & \charis \textepsilon & \charis f
       & \charis \textscriptg & \charis x & \charis i & \charis j & \charis k & \charis l \\ \hline \hline
  リパーシェ & \liparxem & \liparxen & \liparxeo & \liparxep & \liparxer & \liparxes & \liparxet
             & \liparxeu & \liparxev & \liparxew & \liparxex & \liparxey & \liparxez \\ \hline
  転写 & m & n & o & p & r  & s & t & u & v & w & x & y & z \\ \hline
  音素 & \charis m & \charis n & \charis \textopeno & \charis p & \charis r & \charis z & \charis t
       & \charis u & \charis v & \charis w & \charis \textesh & \charis y & \charis \texttslig \\ \hline
 \end{tabular}
\end{center}
\end{table}
以後,転写表記にて記述する.

\subsection{異音}
表\ref{vis_phonology}に発音を示したが,実際には次のような自由異音が認められる.
\begin{itemize}
 \item a: {\charis \textscripta}
 \item e: {\charis e}
 \item g: {\charis \textgamma, \textinvscr, \textscr}
 \item h: {\charis h,\textchi}
 \item i: {\charis \i, \textbari}
 \item r: {\charis \textfishhookr, \textturnr}
 \item u: {\charis \textbaru}
 \item x: {\charis \textctc}
\end{itemize}

\section{文法}
\subsection{品詞}
現代ヴィッセンスタンツ語には次の品詞が存在する.
\begin{itemize}
 \item 名詞
 \item 代名詞
 \item 動詞
 \item 形容詞
 \item 副詞
 \item 前置詞
 \item 接続詞
 \item 関係詞
 \item 数詞
\end{itemize}

\subsection{語順}
現代ヴィッセンスタンツ語の基本語順はSVOである.

\subsubsection{平叙文}
平叙文はほほ例外なくSVOとなる.
\begin{enumerate}
 \item cai hatro kacce. 「私は本を読む.」 
\end{enumerate}

\subsubsection{疑問文}
疑問文の場合,次のようになる.
\begin{enumerate}
 \item hatro tif kacce? 「あなたは本を読むか?」
\end{enumerate}

\subsubsection{否定文}
否定文の場合,次のようなSV ni Oの形式となる.
\begin{enumerate}
 \item cai hatro ni kacce. 「私は本を読まない.」
\end{enumerate}

\subsubsection{所有文・存在文}
所有文及び存在文に関しては否定文が2つ存在する.

所有文では次のようになる.
\begin{enumerate}
 \item cai vooz ni kacce. 「私は本を持っていない.」 \label{ni_neg}
 \item cai vooz kacce met. 「私は本を持っていない.」\label{met_neg}
\end{enumerate}
項\ref{met_neg}は直訳すると「私は0つの本を持っている.」となる.

この2つの文は,項\ref{ni_neg}が「所有するということをしていない」となり
「その状態にする気がない」の意味合いを持つのに対し
項\ref{met_neg}は単に「現時点の状態ではそうである」の意味合いのみという違いがある.

存在文では次のようになる.
\begin{enumerate}
 \item fer gel ni e deef. 「彼(女)は家にいない.」\label{ni_neg_gel}
 \item fer gel e deef nict. 「彼(女)は家にいない.」\label{nict_neg_gel}
\end{enumerate}
項\ref{nict_neg_gel}は直訳すると「彼(女)は家ではないところにいる.」となる.

項\ref{ni_neg_gel}は所有文の項\ref{ni_neg}と同様に「その場所に存在するということをしていない」となり
「あえてその場所にいないようにしている」の意味合いを含む.
項\ref{nict_neg_gel}は口語では単に「その場所にはいない」「別のところにいる」の意味合いとなる.

\subsection{名詞}
ヴィッセンスタンツ語において名詞は孤立語的であり,複数はiweで修飾することで表す.
しかし,一部の単語においてはiwe変化と呼ばれるものが見られる.
\begin{itembox}[l]{iwe変化の一例}
 \begin{center}
  nys $\xrightarrow{\text{複数化}}$ nys iwe $\xrightarrow{\text{iwe変化}}$  nyswe (nys:「女性」) \\
  galpe $\xrightarrow{\text{複数化}}$ galpe iwe $\xrightarrow{\text{iwe変化}}$ galpye (galpe:「子供」)
 \end{center}
\end{itembox}
この変化は日常的に使用頻度の高い語彙によく見られる.

\subsection{代名詞}
現代ヴィッセンスタンツ語の代名詞には性別による違いや活用は存在せず,単複の区別のみが存在する.

\subsubsection{人称代名詞}
現代ヴィッセンスタンツ語の人称代名詞は表\ref{vic_human_pronoum}のようになっている.
\begin{table}[htbp]
 \caption{現代ヴィッセンスタンツ語の人称代名詞表}
 \label{vic_human_pronoum}
 \begin{center}
  \begin{tabular}{|c||c|c|c|} \hline
   & 一人称 & 二人称 & 三人称 \\ \hline \hline
   単数 & cai & tif & fer \\ \hline
   複数 & caiwe & tifie & feriwe \\ \hline
  \end{tabular}
 \end{center}
\end{table}

\subsubsection{指示代名詞}
現代ヴィッセンスタンツ語の指示代名詞は表\ref{vic_obj_pronoum}のようになっている.
\begin{table}[htbp]
 \caption{現代ヴィッセンスタンツ語の指示代名詞表}
 \label{vic_obj_pronoum}
 \begin{center}
  \begin{tabular}{|c||c|c|c|} \hline
   & 近称 & 遠称 & 示称 \\ \hline \hline
   単数 & hom & dul & xip \\ \hline
   複数 & homie & duliwe & xipie \\ \hline
  \end{tabular}
 \end{center}
\end{table}

ここでいう示称とは「会話の中で初めて登場したもの」や「聞き手の認識の範囲にないと話し手が考えるもの」を表すものである.
通常,示称で表される対象を指し示すなどの明示する動作が伴う.

\subsection{動詞}
動詞も名詞と同様に孤立語的であり,時制や相などは副詞で表される.
\begin{itemize}
 \item hatro (読む), hatro fano (何度も読む)
 \item 
\end{itemize}
以下未作成

\subsection{形容詞}
\subsection{副詞}
\subsection{前置詞}
\subsection{接続詞}
\subsubsection{並列接続詞}
\subsubsection{従属接続詞}
\subsection{関係詞}

\end{document}
