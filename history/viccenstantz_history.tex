\documentclass[a4paper,xelatex,ja=standard]{bxjsarticle}

\usepackage{amsmath, amssymb}
\usepackage{hyperref}

\hypersetup{pdfborder={0 0 0},bookmarksopen=true}

\begin{document}

{
\title{\Huge ヴィッセンスタンツ史概論}
\author{asfonont holbazim/渡久地信之}
\date{2017年}
}
\maketitle
\thispagestyle{empty}

% 表紙のページのページ数を消して次からカウントさせるために
% 目次のページの番号をromanにすることで対処している
% 正しいやり方を見付け次第修正が必要かもしれない

\newpage

\pagenumbering{roman}
\tableofcontents

\newpage
\pagenumbering{arabic}

\section{概要}
本資料ではヴィッセンスタンツ史概論を扱う.\\
ヴィッセンスタンツ史は現代までに大きく分けて次の4つに分けることが出来るとされている.
\begin{itemize}
 \item 建国期 (phil.120頃-phil.430頃)
 \item 統治期 (phil.430頃-phil.800頃)
 \item Xelken支配期 (phil.800頃-phil.2003/2月)
 \item 連邦期 (phil.2003/2月-現代)
\end{itemize}

\section{建国期}
\subsection{建国初期}
ヴィッセンスタンツ人の祖先は,大陸ヴィッセンスタンツの東に存在する暗い森よりもさらに東側から来た民族である.
おおよそphil.120頃には,現在のデイシェスのアルマート川河口東側付近にヴィッセンスタンツ王国を
建国しデイシェス周辺を統治していた.


\end{document}
