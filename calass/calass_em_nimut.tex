\documentclass[uplatex,a4paper]{jsarticle}

\usepackage{amsmath, amssymb}

\begin{document}

{
\title{\Huge サラス人の起源}
\author{Skarsna.haltxeafis/渡久地信之}
\date{2017年}
}
\maketitle
\thispagestyle{empty}

% 表紙のページのページ数を消して次からカウントさせるために
% 目次のページの番号をromanにすることで対処している
% 正しいやり方を見付け次第修正が必要かもしれない

\newpage

\pagenumbering{roman}
\tableofcontents

\newpage
\pagenumbering{arabic}

\section{概要}
本論文ではデュインのサラス地方に居住しているサラス人(現地名:``salasem nimut'')の起源について考察する.

サラス人に関して,現地では「大昔,自分達の祖先は別の世界からやってきた」という伝承が残っているが
サラス人のほとんどはネートニアーであることが分かっておりこの伝承の信憑性は低いと考えられている.
一方で,サラス人の話すサラス語(現地名:``salasem qeras'')やサラス人の宗教や文化は
デュインやその周辺地域では確認されておらずその起源は不明のものとなっている.

\section{サラス地方}
\subsection{サラスの気候}
\subsection{サラスの文化}
\subsection{サラスで話される言語}

\subsection{サラスの種族構成}
サラスの種族構成は表のようになっている.
\begin{table}[htbp]
 \begin{tabular}{llr}
  種族 & 人種 & 比率 \\ \hline \hline
  ネートニアー & サラス人 & 59.2\% \\
   & ヴィッセンスタンツ人 & 18.3\% \\
   & その他 & 7.4\% \\ \hline
  人型ラーデミン & サラス人 & 9.6\% \\
   & その他 & 2.7\% \\ \hline
  その他 & & 2.8\%
 \end{tabular}
\end{table}

\end{document}